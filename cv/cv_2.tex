%%%%%%%%%%%%%%%%%%%%%%%%%%%%%%%%%%%%%%%%%
% Medium Length Graduate Curriculum Vitae
% LaTeX Template
% Version 1.1 (9/12/12)
%
% This template has been downloaded from:
% http://www.LaTeXTemplates.com
%
% Original author:
% Rensselaer Polytechnic Institute (http://www.rpi.edu/dept/arc/training/latex/resumes/)
%
% Important note:
% This template requires the res.cls file to be in the same directory as the
% .tex file. The res.cls file provides the resume style used for structuring the
% document.
%
%%%%%%%%%%%%%%%%%%%%%%%%%%%%%%%%%%%%%%%%%

%----------------------------------------------------------------------------------------
%	PACKAGES AND OTHER DOCUMENT CONFIGURATIONS
%----------------------------------------------------------------------------------------

\documentclass[margin, 10pt]{res} % Use the res.cls style, the font size can be changed to 11pt or 12pt here

\usepackage{helvet} % Default font is the helvetica postscript font
%\usepackage{newcent} % To change the default font to the new century schoolbook postscript font uncomment this line and comment the one above

\setlength{\textwidth}{5.1in} % Text width of the document

\usepackage[italian]{babel}
\usepackage[T1]{fontenc}

\usepackage[utf8]{inputenc}

\begin{document}

%NON si mette mai la foto sul curriculum 

%----------------------------------------------------------------------------------------
%	NAME AND ADDRESS SECTION
%----------------------------------------------------------------------------------------

\moveleft.5\hoffset\centerline{\large\bf Federico Belliardo} % Your name at the top
 
\moveleft\hoffset\vbox{\hrule width\resumewidth height 1pt}\smallskip % Horizontal line after name; adjust line thickness by changing the '1pt'
 
\moveleft.5\hoffset\centerline{Lungarno Pacinotti 51} % Your address
\moveleft.5\hoffset\centerline{Pisa, Italy 56126}
\moveleft.5\hoffset\centerline{+393407524606}
\moveleft.5\hoffset\centerline{federico.belliardo@sns.it \textit{or} federico.belliardo@gmail.com}


%----------------------------------------------------------------------------------------

\begin{resume}

%----------------------------------------------------------------------------------------
%	OBJECTIVE SECTION
%----------------------------------------------------------------------------------------
 
\section{OBJECTIVE}  

Obtaining a position at Oxford as a summer student, in order to deepen my knowledge in the field of Physics.

%Obtaining a position at CERN as a summer student, in order to deepen my knowledge %in the field of particle physics.

%----------------------------------------------------------------------------------------
%	EDUCATION SECTION
%----------------------------------------------------------------------------------------

\section{EDUCATION}

{\sl Univesity of Pisa}:   \hfill 2014-present \\
Current undegraduate student in Physics.

{\sl Scuola Normale Superiore di Pisa}: \hfill 2014-present \\
Attending \textit{Corso Ordinario}, Classe di Scienze.
Students of Scuola Normale Superiore (http://www.sns.it/) live together in colleges and have to pass a very selective entrance exam, maintain high grades and attend extra courses on advanced topics.

{\sl I.I.S. "G.Vallauri", Fossano}:  \hfill 2009-2014 \\
Liceo Scientifico-Tecnologico (High school).\\ 
\textit{Diploma di maturit\`a}, marks: 100/100.

%Ho controlato e non mi avevano dato la lode

 
%----------------------------------------------------------------------------------------
%	COMPUTER SKILLS SECTION
%----------------------------------------------------------------------------------------

\section{COMPUTER \\ SKILLS} 

%Dubbio sulla formattazione: mettere in italic languages and operating systems oppure mettere in italics i sistemi e i linguaggi, cioè le cose sostanziali?

{\sl Languages \& Software:} 
%Vedere se mettere questi due.
Python, C, \LaTeX \\
%Spero di imparare mathematica, il c++ non lo ricordo benissimo, ovviamente non lo metto, ma ho un vecchissimo ECDL
{\sl Operating Systems:} Linux, Windows

%----------------------------------------------------------------------------------------
%	LANGUAGES
%----------------------------------------------------------------------------------------

\section{LANGUAGES} 

{\sl First language:} Italian\\
{\sl Others:} English (B2 level, Cambridge English: FCE - 79/100)
%Non vale la pena di metter il pet con 100, la rockehouse che mi ha fatto il test a inizio Normale vale qualcosa?

%----------------------------------------------------------------------------------------
%	PROFESSIONAL EXPERIENCE SECTION
%----------------------------------------------------------------------------------------

 
\section{EXPERIENCES}

%Non ho messo Google Science Fair, anche se sicuramente è conosciuta
%Ho partecipato anche l'anno prima a Zero Robotics ma non siamo andati in finale.

%Non ho messo le provinciali di fisica e le provinciali di matematica

{\sl International Physics Olympiads (IPhO) 2014 held in Astana} \hfill July 2014\\
Silver Medal winner. IPhO is an international event where the best students from all over the world challenge each others in solving theoretical and experimental problems and get the opportunity to meet international students and visit the foreign country in which Olympiads are held.

{\sl Italian Physics Olympiads 2014} \hfill April 2014\\
Gold Medal and pre-IPhO stage at ICTP (\textit{Abdus Salam International Centre for Theoretical Physics}) in Trieste (Italy). Physics competition among the best Italian students. 

{\sl International Physics Olympiads (IPhO) 2013 held in Copenhagen} \hfill July 2013\\
Bronze Medal winner.

{\sl Italian Physics Olympiads 2013} \hfill  April 2013\\
Gold Medal and pre-IPhO stage at ICTP, Trieste.

{\sl Zero Robotics High School Competition 2012} \hfill January 2013\\
Team ROBOVALL - European Agency host at ESA (\textit{European Space Agency}), Leiden (Netherlands).\\
In two words our aim was to program a SPHERE Satellite in order to move and do certain jobs (like taking up objects or shining lasers) with more efficiency than the other teams. We reached the European Finals where our code was run on actual satellite on the ISS (\textit{International Space Station}).

{\sl Italian Physics Olympiads 2012} \hfill April 2012\\
Bronze Medal and pre-IPho stage at ICTP, Trieste.

%Non ho messo le varie scuole estive di fisica sperimentale

%----------------------------------------------------------------------------------------
%	PUBLICATIONS
%---------------------------------------------------------------------------------------- 

\section{PUBLICATIONS} 

Sara Journal (\textit{Journal of the Society of Amateur Radio Astronomers}), March-April 2012, \textit{Application of Empirical Mode Decomposition to the Treatment 
of SID Monitor Data}, Giancarlo Fissore and Federico Belliardo. Presented at \textit{Google Science Fair 2011}.\\
%----------------------------------------------------------------------------------------
%	COMMUNITY SERVICE SECTION
%---------------------------------------------------------------------------------------- 

%\section{COMMUNITY \\ SERVICE} 

%Purtroppo non ho fatto volontariato

%----------------------------------------------------------------------------------------
%	EXTRA-CURRICULAR ACTIVITIES SECTION
%----------------------------------------------------------------------------------------


%----------------------------------------------------------------------------------------
%	EXTRA-CURRICULAR ACTIVITIES SECTION
%----------------------------------------------------------------------------------------

%\section{OTHER ACTIVITIES} 

%Aggiungere eventualmente sport e attività extracurriculari
%Teatro, sport, science fiction e altra letteratura che amo. Non è bene inserire attività che non siano di natura sociale, una cosa da inserire sarebbe tipo se ho fatto teatro (seriamente,...) se sono stato capitano di una squadra di calcio,... I miei obbi sono sempre stati molto solitari. Posso dire che mi piace andare a correre e fare trekking (in reltà non è molto sociale ed è abbastanza inutile in un curriculum come questo, di carattere accademico)

%----------------------------------------------------------------------------------------
%	REFERENCES
%----------------------------------------------------------------------------------------

%Nome e contatto dei professori che hanno scritto le lettere di raccomandazione. Esempio:

%\section{REFERENCES}

%Prof. Luigi (Gigi) Rolandi \hfill Prof. Giovanni Batignani\\
%Scuola Normale Superiore Pisa \hfill Università di Pisa\\
%Experimental Particle Physics \hfill Experimental Particle Physics\\
%Piazza dei Cavalieri 7 \hfill Pisa, Largo Pontecorvo  3\\
%56100 – Pisa – Italy \hfill 56100 – Pisa – Italy\\
%luigi.rolandi@sns.it \hfill giovanni.batignani@unipi.it\\

%Non c'è veramente dibosgno di info biografiche (dove sono nato, dove ho vissuto, parenti,..)

\end{resume}

\end{document}
