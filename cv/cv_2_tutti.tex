%%%%%%%%%%%%%%%%%%%%%%%%%%%%%%%%%%%%%%%%%
% Medium Length Graduate Curriculum Vitae
% LaTeX Template
% Version 1.1 (9/12/12)
%
% This template has been downloaded from:
% http://www.LaTeXTemplates.com
%
% Original author:
% Rensselaer Polytechnic Institute (http://www.rpi.edu/dept/arc/training/latex/resumes/)
%
% Important note:
% This template requires the res.cls file to be in the same directory as the
% .tex file. The res.cls file provides the resume style used for structuring the
% document.
%
%%%%%%%%%%%%%%%%%%%%%%%%%%%%%%%%%%%%%%%%%

%----------------------------------------------------------------------------------------
%	PACKAGES AND OTHER DOCUMENT CONFIGURATIONS
%----------------------------------------------------------------------------------------

\documentclass[margin, 10pt]{res} % Use the res.cls style, the font size can be changed to 11pt or 12pt here

\usepackage{helvet} % Default font is the helvetica postscript font
%\usepackage{newcent} % To change the default font to the new century schoolbook postscript font uncomment this line and comment the one above

\setlength{\textwidth}{5.1in} % Text width of the document

\begin{document}

%NON si mette mai la foto sul curriculum 

%----------------------------------------------------------------------------------------
%	NAME AND ADDRESS SECTION
%----------------------------------------------------------------------------------------

\moveleft.5\hoffset\centerline{\large\bf Federico Belliardo} % Your name at the top
 
\moveleft\hoffset\vbox{\hrule width\resumewidth height 1pt}\smallskip % Horizontal line after name; adjust line thickness by changing the '1pt'
 
\moveleft.5\hoffset\centerline{Lungarno Pacinotti 51} % Your address
\moveleft.5\hoffset\centerline{Pisa, Italy 56126}
\moveleft.5\hoffset\centerline{3407524606}
\moveleft.5\hoffset\centerline{federico.belliardo@sns.it \textit{or} federico.belliardo@gmail.com}


%----------------------------------------------------------------------------------------

\begin{resume}

%----------------------------------------------------------------------------------------
%	OBJECTIVE SECTION
%----------------------------------------------------------------------------------------
 
\section{OBJECTIVE}  

Obtaining a position at DESY as a summer student, in order to deepen my knowledge in the field of Physics.

%----------------------------------------------------------------------------------------
%	EDUCATION SECTION
%----------------------------------------------------------------------------------------

\section{EDUCATION}

{\sl Univesity of Pisa}:   \hfill 2014-present \\
Current undegraduate student in Physics.

{\sl Scuola Normale Superiore di Pisa}: \hfill 2014-present \\
Attending \textit{Corso Ordinario}, Classe di Scienze.

{\sl I.I.S. "G.Vallauri", Fossano}:  \hfill 2009-2014 \\
Liceo Scientifico-Tecnologico (High school).\\ 
\textit{Diploma di maturit\`a}, marks: 100/100.

%Ho controllato e non mi avevano dato la lode

 
%----------------------------------------------------------------------------------------
%	COMPUTER SKILLS SECTION
%----------------------------------------------------------------------------------------

\section{COMPUTER \\ SKILLS} 

%Dubbio sulla formattazione: mettere in italic languages and operating systems oppure mettere in italics i sistemi e i linguaggi, cioè le cose sostanziali?

{\sl Languages \& Software:} 
%Vedere se mettere questi due.
Phyton, C, Latex\\
%Spero di imparare mathematica, il c++ non lo ricordo benissimo, ovviamente non lo metto, ma ho un vecchissimo ECDL
{\sl Operating Systems:} Linux, Windows

%----------------------------------------------------------------------------------------
%	LANGUAGES
%----------------------------------------------------------------------------------------

\section{LANGUAGES} 

{\sl First language:} Italian\\
{\sl Others:} English (B2 level, Cambridge English: FCE - 79/100)
%Non vale la pena di metter il pet con 100, la rockehouse che mi ha fatto il test a inizio Normale vale qualcosa?

%----------------------------------------------------------------------------------------
%	PROFESSIONAL EXPERIENCE SECTION
%----------------------------------------------------------------------------------------

 
\section{EXPERIENCES}

%Non ho messo Google Science Fair, anche se sicuramente è conosciuta
%Ho partecipato anche l'anno prima a Zero Robotics ma non siamo andati in finale.

%Non ho messo le provinciali di fisica e le provinciali di matematica

{\sl International Physics Olympiads (IPhO) 2014 held in Astana} \hfill July 2014\\
Silver Medal winner.

{\sl Olimpiadi italiane della Fisica 2014 - Fase Nazionale} \hfill April 2014\\
Gold Medal and pre-IPho stage at ICTP (\textit{Abdus Salam International Centre for Theoretical Physics}) in Trieste (Italy).

{\sl International Physics Olympiads (IPhO) 2013 held in Copenhagen} \hfill July 2013\\
Bronze Medal winner.

{\sl Olimpiadi italiane della Fisica 2013 - Fase Nazionale} \hfill  April 2013\\
Gold Medal and pre-IPho stage at ICTP, Trieste.

{\sl Zero Robotics High School Competition 2012} \hfill January 2013\\
Team ROBOVALL - European finalist in ISS Competition (\textit{International Space Station}), host at ESA (\textit{European Space Agency}), Leiden (Netherlands).

{\sl Olimpiadi italiane della Fisica 2012 - Fase Nazionale} \hfill April 2012\\
Bronze Medal and pre-IPho stage at ICTP, Trieste.

%Non ho messo le varie scuole estive di fisica sperimentale

%----------------------------------------------------------------------------------------
%	PUBLICATIONS
%---------------------------------------------------------------------------------------- 

%----------------------------------------------------------------------------------------
%	COMMUNITY SERVICE SECTION
%---------------------------------------------------------------------------------------- 

%\section{COMMUNITY \\ SERVICE} 

%Purtroppo non ho fatto volontariato, ma non vale la pena di metterlo

%----------------------------------------------------------------------------------------
%	EXTRA-CURRICULAR ACTIVITIES SECTION
%----------------------------------------------------------------------------------------

%\section{OTHER ACTIVITIES} 

%Aggiungere eventualmente sport e attività extracurriculari
%Teatro, sport, science fiction e altra letteratura che amo. Non è bene inserire attività che non siano di natura sociale, una cosa da inserire sarebbe tipo se ho fatto teatro (seriamente,...) se sono stato capitano di una squadra di calcio,... I miei hobbi sono sempre stati molto solitari. Posso dire che mi piace andare a correre e fare trekking (in reltà non è molto sociale ed è abbastanza inutile in un curriculum come questo, di carattere accademico)

%----------------------------------------------------------------------------------------
%	REFERENCES
%----------------------------------------------------------------------------------------

%Nome e contatto dei professori che hanno scritto le lettere di raccomandazione. Esempio:

\section{REFERENCES}

Prof. Luigi (Gigi) Rolandi \hfill Prof. Giuseppe Carlo La Rocca\\
Scuola Normale Superiore Pisa \hfill Scuola Normale Superiore Pisa\\
Experimental Particle Physics \hfill Physics of Matter\\
Piazza dei Cavalieri 7 \hfill Piazza dei Cavalieri 7\\
56100 – Pisa – Italy \hfill 56100 – Pisa – Italy\\
gigi.rolandi@sns.it \hfill giuseppe.larocca@sns.it\\

%----------------------------------------------------------------------------------------

%----------------------------------------------------------------------------------------
%	MOTIVATIONAL LETTER
%----------------------------------------------------------------------------------------
\section{DESY LETTER}


My interest in natural sciences has always been strong, now that I'm an undergraduate student I want to gain insight in the field of physics research and I feel that
I need some hand on experience in order to truly understand if research is what I want to do in my life. I'm curious to learn more about what is done in Desy's labs and I'm undecided what field of 
physics purse in my future studies: whether to do be an experimentalist or a theoretician, whether to dive into high or low energy physics. I think Desy will give me the opportunity to get in contact with modern matter related and particles science. I have good programming skills that I developed during high school, so I would also enjoy a computational related project.
Moreover I will surely be enriched by the contact with international students and by the experience of living abroad for a while in the city of Hamburg.


%----------------------------------------------------------------------------------------
%	BIOGRAPHICAL INFORMATION
%----------------------------------------------------------------------------------------

%Non c'è veramente bisogno di info biografiche (dove sono nato, dove ho vissuto, parenti,..)

\end{resume}

\end{document}
