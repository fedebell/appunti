\documentclass[10pt,a4paper]{article}
\usepackage[utf8]{inputenc}
\usepackage[italian]{babel}
\usepackage{amsmath}
\usepackage{amsfonts}
\usepackage{amssymb}
\usepackage{hyperref}
\usepackage{graphicx}
\usepackage[left=2cm,right=2cm,top=2cm,bottom=2cm]{geometry}
\newcommand{\rem}[1]{[\emph{#1}]}

\author{Belliardo Federico}
\title{Note sugli appunti di relatività di Reall}
\begin{document}
\maketitle

\begin{itemize}

\item p.34 Fare esercizio alla fine della pagina.
\item p.36 La derivazione della geodetica funziona anche nel caso di fotoni? Non credo perché in quel caso $G = 0$ sempre e on può stare a denominatore dell'equazione geodetica.
\item p.38 Cose si può vedere che in generale la geodesica massimizza il tempo proprio?
\item p.38 La lagrangiana non dovrebbe essere scritta in termini di $\tau$ ma in termini di un parametro $\lambda \in [0, 1]$: $\mathcal{L} = \sqrt{-g_{\mu \nu}(x(\lambda)) \frac{d x^{\mu} (\lambda)}{d \lambda}  \frac{d x^{\nu} (\lambda)}{d \lambda}}$. Si osservi poi che la lagrangiana non ha una dipendenza esplicita da $\lambda$: $\frac{\partial \mathcal{L}}{\partial \lambda} = 0$ questo implica l'esistenza di uan quantità conservata (l'Hamiltoniana): se la si calcola si vede che $-g_{\mu \nu}(x(\lambda)) \frac{d x^{\mu} (\lambda)}{d \lambda}  \frac{d x^{\nu} (\lambda)}{d \lambda}$ deve essere conservata sul moto lungo una geodetica (verificare!)
\item p.38 Non mi è chiaro perchè si possa scrivere la lagrangiana ed eseguire la variazione in termini del tempo proprio? Non serve un parametro aggiuntivo?
\item p.41 Mostrare che la differenza di due connessioni è un tensore (1, 2).
\item p.45 Non mi convince troppo la derivazione di $\nabla_{Y}  Y = fY$, non dovrebbe essere $h:M \rightarrow \mathbb{R}$? Qui sembra che sia  $h:\mathbb{R} \rightarrow \mathbb{R}$
\item p.46 Postulato della relatività generale è che le particelle si muovono su geodetiche con parametrizzazione affine, cioè su linee integrali di campi vettoriali che soddisfano $\nabla_{X} X = 0$. Da questa equazione si può recuperare l'equazione geodetica prima ottenuta con la lagrangiana. Vi è una riparametrizzazione lineare del tempo affine che ha dunque due parametri liberi, che diventano uno del caso di curve time-like o space-like, imponendo la condizione $g(X, X) = \pm 1$. Nel caso di lice non esiste un tempo proprio ma ho due gradi di libertà nella parametrizzazione. $\nabla_{X} g(X, X) = 0$ Dunque il modulo della quadrivelocità è fissato (e uguale a -1) nel caso di temo proprio. 
\item p.47 \textbf{Coordiante normali Riemanniane}:
Dato un punto di una varietà costruisco la geodetica che passa per quel punto e che è tangente al vettore dato risolvendo le opportune equazioni geodetiche. Il tempo affine vale 0 sul punto p. Vi è una libertà di scala nel tempo affine. Localmente   la mappa esponenziale è un isomorfismo e può essere usata per definire un sistema di coordinate normale. Come si fa a fissare consistemente la velocità di tutte le geodetiche.
Alla fine della dimostrazione si fa vedere che $\Gamma^{\mu}_{\rho \sigma} k^{\rho} k^{\sigma} = 0 \rightarrow \Gamma^{\mu}_{[\rho \sigma]} = 0$: questo perché prendendo $k$ in modo che siano uguali a 1 solo le coordinate corrispondenti al simbolo che vogliamo annullare si vede che segue la tesi. Vanno bene vettori iniziali ti qualunque tipo (anche non fisici).
\textbf{Vedersele su un libro di geometria, come l'Abate, riportare la dimostrazione di Hawking and Ellis, molto più concisa e apprezzabile}.

\item p.48 Non mi è chiaro perchè in un sistema di coordiante normale il tensore metrico è: $g_{\mu \nu} = \eta_{\mu \nu}$

\item p.54 Dimostrare equazioni di Navier-Stokes relativistiche.

\item p.57 Non capisco dove va a finire la derivata rispetto al  commutatore nell'identità di Ricci.

\item p.58 Dimostrare che se due campi vettoriali X e Y commutano si può trovare un sistema di coordinate in cui $X = \frac{\partial}{\partial s}$ e $Y = \frac{\partial}{\partial t}$. \textbf{Importante da FARE!}

\textbf{Appunti sulla deviazione geodetica}
Sia data la mappa $\gamma: I \times I' \rightarrow M$ questa definisce una carta in cui le due coordinate della sottovarità sono $(s, t)$. Sia $\gamma(t, s)$ una curva geodetica con t parametro affine per ogni s. Costruisco il campo vettoriale $T$ definendolo in un punto $(s, t)$ come il vettore tangente alla geodetica passante per s e $S$ è il campo vettoriale in ogni punto tangente alla curva $\gamma_{t}(s)$. Per definizione questi due campi vettoriali sono: $S = \frac{\partial}{\partial s}$ e $T = \frac{\partial}{\partial t}$. Essendo vettori di base relativi alle coordinate normali commutano. Si può anche vedere che essendo le loro componenti legate alla derivata dela funzione $\gamma$ in un qualunque sistema di riferimento ho che il commutatore è nullo (se lo scrivo esplicitamente). Come si può anche vedere applicando la definizione in una base generica conoscendo componenti $T^{\mu} = \frac{\partial x^{\mu}(x, t)}{\partial t}$. Vale la relazione: $x^{\mu} (t, s+ds) = x^{\mu} (t, s+ds) + ds S^{\mu} + O(ds^2)$ per definizione stessa di vettore tangente. Quindi $\nabla_{T} S$ quantifica la velocità di allontanamento di due geodetiche mentre $\nabla_{T}\nabla_{T} S$ ne quantifica l'accelerazione. 

Il tensore torsione è definito come: $\mathbf{T} (T, S) = \nabla_{T} S - \nabla_{S} T - [T, S]$. Dunque $\nabla_{T} S = \nabla_{S} T$ e quindi   $\nabla_{T}\nabla_{T} S = \nabla_{T}\nabla_{S} T$ = $\nabla_{S}\nabla_{T} T + \mathbf{R}(T,S)T = \mathbf{R}(T,S)T$. L'equazione geodetica è:  $\nabla_{T}\nabla_{T} S = \mathbf{R}(T,S)T$.

Caveat: la parametrizzazione che abbiamo dato è solamente bidimensionale. E' ncessario introdurre tanti vettori deviazioni quante sono le dimensioni del possibile "fascio" di geodetiche che costruiamo. Così che tutti i campi vettoriali che studio siano definiti in un intorno aperto (a parte interna non nulla) della varietà spazio-tempo. Si può anche studiare come cambia il volume di un "bunch" di geodetiche che evolvono vicine, vedi Carroll e Baez ("The meaning of Einstein Equation").

\item p.58 Rivedere calcolo della non commutatività delle derivate covarianti rispetto a campi vettoriali a commutatore nullo, per mostrare che il trasporto parallelo su loop chiuso seguendo le geodetiche di due diversi campi vettoriali commutanti è legato al tensore di Riemann.

\item p.62 Svolgere gli esercizi sul tensore di Riemann.

\item p.63 Dimostrare la contracted Bianchi identity.
\item p.64 Approfondire il teorema di Lovelock.

\item p.68/69 Eseguire i calcoli per trovare le componenti di generico tensore in seguito a pullback e pushforward.

\textbf{Nota sui tensori}
Un covettore è definito come una funzione lineare $TM \rightarrow \mathbb{R}$, mentre in virtù dell'isomorfismo canonico con il biduale si può definire un vettore come una funzione $TM^{*} \rightarrow \mathbb{R}$. Un tensore è definito come una funzione $TM \times TM ... TM^{*} \times TM^{*} \rightarrow \mathbb{R}$ lineare. Questa è la giusta definizione di tensore dalla geometria. Ma come si lega con la definione algebrica mediante la proprietà universale?

\textbf{Finire}: le due definizioni che ho in mente probabilmente si legano grazie alla proprietà universale.

\item p.69 Dimostrare che i pullback e pushforward commutano con la contrazione e il prodotto tensore.

\item p.72 Dimostrare esplicitamente che la mappa esponenziale del campo vettoriale X è localmente un isomorfismo.

\item p.72 Mostrare che localmente esiste una superficie che è sempre ortogonale al campo vettoriale X. Una ipersuperficie può essere definita da una funzione $f(x) = cost$ non degenere (attenzione ai teoremi di analisi 3 sulle funzioni inverse) Essa definisce una ipersuperficie. Va da se che il campo vettoriale $\eta^a = g^{ab} \nabla_{b} f$ è ortogonale alla varietà. Viceversa se ho un campo vettoriale posso costruire la 1-forma $g_{ab} X^{b}$, questa localmente (in un intorno semplicemente connesso) può essere integrata per dare luogo localmente a una superficie ortogonale a X.

\item p.72 Calcolare le derivate di Lie del campo di covettori (fatto) e della metrica (da fare)

\item Read linerized theory

\item p.102 Show that the definition of external derivative is indipendet on the chart. Show that pullback and extrnal derivative commute. Read about Poincarè lemma.
\end{itemize}

\end{document}